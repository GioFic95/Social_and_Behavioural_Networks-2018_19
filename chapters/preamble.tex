% !TeX spellcheck = it_IT

\chapter{Preamble}
Questi appunti utilizzano una notazione personale che sto portando avanti da anni, con l'obiettivo di essere il meno ambiguo possibile. Un caso comune è il seguente: per definire una funzione, utilizzo questa sintassi:

\begin{equation}
f \in A \to B : a \mapsto f(a)
\end{equation}

dove $f$ è la funzione, $A \to B$ è il \textbf{tipo} di funzione, specificato dal dominio $A$ e dal codominio $B$, ed è trattato come insieme di tutte le funzioni da $A$ a $B$, giustificando l'uso dell'operatore di appartenenza. La parte che segue i due punti definisce la funzione attraverso la notazione di mappatura.
Ad esempio, l'espressione seguente:

\begin{equation}
d \in \mathbb{N} \to \mathbb{N} : n \mapsto 2\cdot n
\end{equation}

si leggerebbe: "$d$ è una funzione che va dai naturail ai naturali tale che ad n associa 2n"