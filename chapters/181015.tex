% !TEX spellcheck = en-US

\chapter{181015}
	
	First lesson reprise - Flow of memes
	
	spread of memes/viruses from a website to another
	
	so, we get traces
	
	goal is to reconstruct graph from traces
	
	Trace spreading model (from virology)
	
	source chosen \uar and edge traversal time chosen by (usually) Exp(lambda)
	trace starts from source
	
	obs. the first two nodes of a trace are connected
	
	pr a source is chosen: 1/n
	pr the edge's other endpoint: 1/deg(source) because of how traces work
	
	n numero di tracce: $pr = (1-1/deltan)^(3deltanlogn) = e^(-3lnn) simeq 1/n^3$
	
	
\subsection{another information spreading process}
	
	NPR chain letter
	
	important observations: first name is fake, the others are plausibly real
	
	asks to add own's name, and then forward to all friends
	
	%%%%%%%%%%%%%%%%%
	
	someone breaks the rules, and decides to publish their copy of the mail
	
	those published mails reveal a subtree of the whole chainletter tree
	
	revealed tree MUCH smaller, and extended in-depth
	
	%%%%%%%%%%%%%%%%%
	
	Let T be a tree, each node has a chance of being "exposed" by a probability p, and if so, reveals all ancestors
	
	estimate the size of the tree
	
	ALGO
	throw a coin on all the nodes of the chain tree, don't throw if node is parent of a revealed node (no need) := special nodes
	
	special nodes are all iid from exposed ones; delta = n of special nodes/special nodes; revealed nodes = delta dot n (it's the expected value!!) -> from there estimate n
	
	delta is the actual exposure probability
	
	the two estimates can go wrong...
	
	examples in nature suggest some bounds
	
	%%%%%%%%%%%%%%%%%%%%%%%
	
	Theorem: the algorithm correctly estimates n if $n > omega(max(1/delta^2, k/delta))$
	
		
		IMPORTANT: Estimates can be redone, but reconstructing another revealed tree is impractical, so we're estimating a characteristic of  a tree by observing a sample revealed one and making estimates
	
	%%%%%%%%%%%%%%%%%%%%%%%
	Single-child fraction
	
	Fix a bound: a node's maximum degree must be lesser than k
	
	partition unknown tree into subforests, each subforest has 1/delta nodes and median height $omega(log_(k-1)^delta^-1)$
	
	by exposing a node in a subforest's lower half, we're exposing a number of nodes equal to the sf's median height
	
	
	
	%%181017%%%%%%%%%%%%%%%%%%
	
	Insight: $\delta$ is unknown too
	
	Recap of revealed subtrees
	
	Exposure prob: $\delta > 0$
	
	$E(\# exposed nodes) = \delta n$
	
	\
	
	Let $x \in RV$ be the set of nodes of the revealed tree. Let $Y \in Coin(p)$, where $Y_v = 1$ if v is exposed, 0 otherwise.
	
	$E(Y_v) = \delta$
	
	Let $ Y = \sum_{v \in X} Y_v \Rightarrow E(Y)= \delta |X|$
	
	OUTPUT: $\hat\delta = \frac{Y}{|X|}$
	
	Chernoff bound is applicable, the Ys are IID
	
	Note: we can do the first substitution because of linearity (?)
	
	"Multiplicative approximation": $\displaystyle \Pr(|\hat\delta - \delta| \geq \varepsilon \delta) = \Pr(|\frac{Y}{|X|} - \delta| \geq \varepsilon \delta) = \Pr(|Y - \delta |X|| \geq \varepsilon \delta |X|) = \Pr(|Y - E(Y)| \geq \varepsilon E(Y)) \leq 2e^{-\frac{\varepsilon^2}{3}E(Y)}, \forall \varepsilon \in (0, 1)$
	
	This is the Multiplicative Chernoff Bound: Let y1 ... yn IID Coins, then (see above, last inequality)
	
	we need E(y) to be large in order to obtain a useful bound, for E close to 1, the bound itself goes over 1, becoming useless
	
	$\displaystyle 2e^{-\frac{\varepsilon^2}{3}E(Y)} = 2e^{-\frac{\varepsilon^2}{3}\delta|X|}$
	
	If $\displaystyle |X| \geq \frac{3}{\varepsilon^2\delta}\ln2/\eta$, then $\displaystyle 2e^{-\frac{\varepsilon^2}{3}\delta|X|} \leq 2e^{-\ln2/\eta} = 2\eta/2 = \eta$
	
	$Y' = \sum_{v \in T}Y_v \Rightarrow E(Y') = \delta n$
	
	$\Pr(Chernoff\ on\ Y') \leq 2e^{-\frac{\varepsilon^2}{3}\delta n}$
	
	Using the bound over $|X|$, then $\leq \eta$
	
	Final output: $\displaystyle \frac{Y'}{\mathaccent 1 \delta } (\leq  \frac{(1 + \varepsilon)\delta n}{(1 - \varepsilon)\delta} = (1+O(\varepsilon))n )$
	
	Lower bound inverts the sign of bigOepsilon
	
	Insight: $\displaystyle \frac{Y'}{\mathaccent 1 \delta } = \frac{Y'}{Y}|X|$
	
	
	
	"An additive approximation is useless"
	
	
	goodnes sof alogrithm:
	
	 - nodes of unknown tree less than delta: revealed tree =0 almost surely
	 - tree is a star -> almost surely get a bad approximation
	 
	what characteristics should the unknown tree have?
	
	Claim: If the number of internal nodes of the revealed tree is at least 3/(epsilonSQRdelta)ln(2/eta) then our guess \^n is going to satisfy (1-oe)n <= \^n <= (1+oe)n with probability at least 1-2eta
	
	the three greek letters can change the goodness of our result
	
	Next goal: find the probability that a node of the unknown tree will be an internal node of the revealed tree
	
	Assumption: The unknown tree is such that none of its nodes has more than K children.
	%That actually has lots of applications! makes much sense
	
	Let $v$ be a node of the unknown tree with K children, then $\Pr($at least 1 children in K is exposed$) \geq 1-\varepsilon^{-1}min(1, \delta K_v)$
	
	$K_v$: expected number of children revealed
	
	Let $Z_v \in Coin(p)$ where it is 1 if at least one child of v is exposed
	
	$\displaystyle E(Z_v) = 1 - (1 - \delta)^{K_v} = 1 - ((1 - \delta)^{1/\delta})^\delta{K_v}$
	
	%using e's limit definition
	$\lim\limits_{\varepsilon \to 0^+}(1-\varepsilon)^{1/\varepsilon} = 1/e$
	
	$1 - ((1 - \delta)^{1/\delta})^\delta{K_v} \geq 1- e^{-\delta K_v}$
	
	
	\begin{itemize}
		\item if deltakv geq 1 then E(Zv) geq 1-1/e
		\item else , geometric intuition... ?????????? ... E(Zv) geq 1-ePOW(-deltakv) geq deltakv(1 - 1/e)
	\end{itemize}
	Lemma is proven
	
	
	\
	
	Former insight: for revelation we care about children, not all of the descendants
	
	By summing all children of all nodes, we get all the edges -> n-1
	% THis is used elsewhere...
	
	\
	
	Lemma: Let Z be the set of nodes of which at least 1 child is exposed; then $\Pr(|Z| \geq 1/2(1-1/e)\min(K^{-1}, \delta)(n-1))\geq 1-e^{-\Theta(n\min(1/K, \delta))}$
	
	Proof: Let I be the set of internal nodes of T; Let D subset I contain nodes with at least 1/delta children; $E(|Z|) = \sum_{v \in I}E(Z_v) \geq (1 - 1/e)\min(1, K_v\delta) = (...)(|D| + \delta\sum_{v \in I-D}K_v)$

	obs1: sumkvforvinI = n-1
	
	obs2: $|D| \geq \frac{\sum_{v \in D}K_v}{|K|}$
	
	then : $(1 - 1/e)(|D| + \delta\sum_{v \in I-D}K_v) \geq (1 - 1/e)(\sum K_v/K + \delta\sum_{v \in I-D}K_v) \geq (...)min(1/K, \delta)\sum = (...)min(...)(n-1)$
	
	Proven
	
	\
	
	see emergency notes!!
	
\section{181022}
	
	-combinatoric optimization-
	
	graph independent set - impractical example
	
	An implicit opt problem.
	
	unknown - partly known function, exp in the codomain
	
	ex: describe SAT by means of truth tables - rows are exponential in variable number
	
	"Modular set functions"
	
	given $X=[n] f \in 2^X \to \mathbb{R}$
	
	$n=3, w_1 = 1, w_2 = 5, w_3 = 1, f(S)=\sum_{i \in S}$
	
	max  set: all domain
	
	max set w neg values: exclude neg values
	
	se ho un oracolo che mi calcola f(s), ed io non conosco f, posso interrogare l'oracolo sui singoletti, e poi applico le logiche precedenti
	
	(Matroid) Ex tutte parti con al più k elementi (a partire da U)
	
	"Ad placement"
	arriva un utente sul sito tipo Google, scegli quali ad mostrare in modo da guadagnare il più possibile
	
	modello Independent Cascade Model
	ogni ad rappresentata da una coppia $AD_i = (v_i, p_i)$ accordo tra pubblicità e Google, $v_i \in \$, p_i$ probobilità di click
	
	$Ad_{samsung} : (1\$, 0.1) \Rightarrow E[X_i] = 1 \cdot 0.1$
	
	Strategy: put highest expectation ads on top
	
	third value,: satisfaction factor (or the will to look other ads)
	
	how to optimize f in ordder to maximize it
	
	caso particolare "submodular set function" - analogo discreto delle funzioni convesse - can be misleading
	
	given X, f is submodular iff $\forall S, T \in X, f(S)+f(T) \geq f(S\cap T) + f(S\cup T)$
	
	modular is submodular: $\forall S, T \in X, f(S)+f(T) = f(S\cap T) + f(S\cup T)$

	quanti bit servono per rappresentare tale funzione?
	
	LAPSE
	
	COVERAGE (function, submodular): $X=\{S_1, S_2...S_m\}, S_i \subseteq [n]$
	
	$Y\subseteq X : c(Y) = |\bigcup_{S_i \in Y}S_i|$
	
	
	$s(Y) = \sum_{S_i \in Y} |S_i|$
	
	
	proof that f (or c?) is submodular
	base case n=1: couple S T can assume 4 combinations: $
	case s, t = \{1\} \Rightarrow union an intersection f(\cdot)= 1	
	1 in T, 1 not in S then 1 in union, 1 not in intersection
	dual is dual
	s, t = \emptyset
	$
	
	induction: split a coverage function into two coverage functions: $c(Y) = |\bigcup_{S_i \in Y}S_i| = |(\bigcup_{S_i \in Y}S_i)-\{n+1\}| + |(\bigcup_{S_i \in Y}S_i) \cap \{n+1\}|$
	
	Nemauser-W
	
	If f is submodular nonnegative and monotone, it can be approx to $max_\{|S|=K\}(f(S))$ to a factor of $1- \frac{1}{e}$
	
	represent modulars as points of convex multidimensional functions
	
	
	"maxcover" data una classe di sottoinsiemi di X,. S-1, si trovino k insiemiinX tail che , insieme, coprano il massinmo numero di elementi
	
	
	Before proving, ALGORITHM (greedy), parameterized by f, X and k: trovare k sottoinsiemi di X che massimizzano f
	
	$
	S_0 <- \emptyset
	for i=1 -> k
		let x_i = maximizer of f(\{x_i\} \cup S_{i-1})
		S_i <- \{x_i\} \cup S_{i-1}		
	return S_k
	$
	
	again before proving, observation: diminishing returns (formalization)
	
	$\delta(x|S) = f(\{x\}\cup S) - f(S)$ %always positif if f is monotone
	
	let f be submodular, $B \subseteq A, x \notin B \Rightarrow \Delta(x|A) \geq \Delta(x|B)$ % an iff can be demonstrated
	
	submodular functions exhibit diminishing returns property
	
	Now: $S=A\cup\{x\}, T=B \Rightarrow f(A\cup\{x\})+f(B)\geq f(A)+f(B\cup\{x\}) \Rightarrow f(A\cup\{x\})-f(A)\geq f(B\cup\{x\})-f(B) \Rightarrow \Delta(x|A) \geq \Delta(x|B)$
	
	proof of approximation:
	
	take an optimal solution $S*=\{x_1^*, ..., x_k^*\}$
	
	$\forall i, f(S^*) \leq f(S^* \cup S_i)$ %per monotonia di f
	
	$ = ... f(S_i) + \sum_{j=1}^{k}\Delta(x_j^*|S_i\cup \{x_1^*, ..., x_{j-1}^*\}) \leq (bydiminishingreturns) f(S_i) + \sum_{j=1}^{k}\Delta(x_j^*|S_i) -> \leq (bygreediness) f(S_i) + \sum_{j=1}^{k}\Delta(x_{i+1}|S_i) = f(S_i) + k(f(S_i \cup \{x_{i+1}\})-f(S_i)) = kf(S_{i+1}) - (k-1)f(S_i)$
	
	
	So: $f(S^*) - f(S_i) \leq k(f(S_{i+1} - f(S_i)))$
	
	Def: $\delta_i = f(S^*) - f(S_i)$
	
	Then: $\delta_i \leq k(\delta_i - \delta{i+1}) \Rightarrow k\delta_{i+1} \leq (k-1)\delta_i \Rightarrow \delta_{i+1} \leq (1-\frac{1}{k})\delta_i$
	
	$\delta_0 = f(S^*)-f(S_0) \leq f(S^*)$ %by nonnegativity
	
	
	.
	.
	.
	
	going to definition of e by its limit form $(1-1/k)^k$
	
	
\section{181024}
	
	recap of yesterday:
	
	given a ground set, and $f \in 2^X \to \mathbb{}$, then f is modular iff $\forall S, T \subseteq X (f(S)+f(T) \geq f(s\cup t) + f(s \cap t))$
	
	E.g.: coverage functions, X is a set system $c(X) = \left| \bigcup_{S \in X} S \right|$

	"given a subset of sets, the coverage value would be the cardinality of the union"
	
	claim: coverage function is submodular
	proof: $c(S)+c(T) \geq c(s\cup t) + c(s \cap t)$
	
	proven case-by-case in the scope of where the elements are inside the subsets
	
	\
	
	if $f_1, ..., f_k \in 2^X \to \mathbb{R} are submodular, \alpha_1, ..., \alpha_k \geq 0 \Rightarrow \sum_{i=1}^{k}\alpha_i f_i is submodular$
	
	proof: the i-th case is easily proven, then sum up; nonnegativity is essential here
	
	\
	
	algorithm for optimization (?): $GREEDY_f(X, K)$:
	$
	S_0 <- \emptyset
	for i=1 -> k
		let x_i = maximizer of f(\{x_i\} \cup S_{i-1})
		S_i <- \{x_i\} \cup S_{i-1}		
	return S_k
	$
	
	if f is submodular, nonnegative and monotone, then $\displaystyle f(S_K)\geq(1-1/e)max_{S \in \binom{X}{K}}f(S)$
	
	\
	
	another algorithm: $GREEDY_{\varepsilon, f}(X, K)$:
	$
	S_0 <- \emptyset
	for i=1 -> k
		%let x_i be : f(S_{i-1} \cup \{x_i\}) \geq (1-\varepsilon)max_{x \in X} f(S_{i-1} \cup \{x\}) %finding the maximizer
		let x_i be : f(S_{i-1} \cup \{x_i\}) - f(S_{i-1}) \geq (1-\varepsilon)max_{x \in X} (f(S_{i-1} \cup \{x\}) - f(S_{i-1})) %finding the maximum marginal return
		S_i <- \{x_i\} \cup S_{i-1}		
	return S_k
	$
	
	the difference is: we don't need to find the exact maximizer, but something (1-epsilon) close to it; and it may be MUCH easier to find such something instead of the actual maximizer.
	
	\
	
	the core of the discussion is approximating a (usually impracticable to compute) function to a (polynomial) one
	
	Tidbit: proving correctness of this algorithm proves the previous one
	
	Trail: KKT model
	
	Hint(!): We don't need to find the maximum in this algorithm (otherwise, we are actually in the former case, why not use it directly?)
	
	\
	
	Theorem: ($\forall \varepsilon \in [0, 1)$), if f is submodular, nonnegative and monotone, then $f(S_k) \geq (1-\varepsilon)(1-1/e)max_{S \in \binom{X}{K}}f(S)$

	Proof: $S^* = argmax\ f(S), S \in \binom{X}{K}$
	
	$\forall i \in \mathbbm{k}$
	
	$f(S^*) \leq f(S_i) + \sum_{j=1}^{k}(f(S_i \cup\{x_1^*, ..., x_j^*\})-f(S_i \cup\{x_1^*, ..., x_{j-1}^*\}))$
	
	[defining diminishing returns (or marginal increase of {1} having {2})]
	
	$\to = f(S_i)+ \sum_{j=1}^{k}\Delta(x_j^* | S_i \cup \{x_1^*, ..., x_{j-1}^*\})$
	
	Because of submodularity, $\to \leq \sum_{j=1}^{k}\Delta(x_j^* | S_i)$
	
	-brainlapse-
	
	...at the end of the day $f(S^*) - f (S_i) \leq \frac{k}{1-\varepsilon}\Delta(x_{i+1}| S_i)$
	
	def $\delta_i = f(S^*) - f(S_i)$
	
	$\delta_{i+1} \leq \delta_i(1-\frac{1-\varepsilon}{k})$
	
	-SIGSEGV-
	
	see previous day to get all deltas from 0 to k
	
	then $f(S_k) \geq (1-(1-\frac{1-\varepsilon}{k})^k)f(S^*)=(1-((1-\frac{1-\varepsilon}{k})^{\frac{k}{1-\varepsilon}})^{1-\varepsilon})f(S^*) \geq (1-(1/e)^{1-\varepsilon})f(S^*)$
	
	more algebretta, use exponential convexity, get $f(S_k) \geq (1-e^{\varepsilon-1})f(S^*)$, over and out
	
	\
	
	Application: Kempe-Kleinberg-Tardos