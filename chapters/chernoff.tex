\newpage
\section{Chernoff bound}

	The Chernoff bound allows you to control how much a RV is close to its expected value.

	Let $X_1, \dots, X_n \in \coin(p)$ \iid, i.e. $X_1, \dots, X_n$ are $n$ RV such that\\
	$\Pr{X_i=1} = p$ and $\Pr{X_i=0}=1-p$.\\
	In other words, they are $n$ independent throws of the same coin.
	
	Let $X=\sum_{i=1}^{n}(X_i)$.
	X is essentially a binomial RV: $X \in \binomdist(\binary^n, p)$, thus $E[X]=p\cdot n$. It represents the number of times you get head with the aforesaid $n$ throws.
	
	So, given an error limit $\varepsilon$, we have the following definition:
    \begin{defn}[Chernoff bound]
       	\begin{equation} \label{chernoff}  %True definition
            \Pr{\abs{X-E[X]} \geq t} \leq e^{-2t^2/n}
        \end{equation}
        where $t := \varepsilon n$.
    \end{defn}
	
	In the binomial case, [\ref{chernoff}] translates to:
	\begin{equation*}
	\Pr{\abs{\sum_{i=1}^{n}(X_i) - p \cdot n} > t} = \Pr{\abs{\frac{1}{n}\sum_{i=1}^{n}(X_i) - p} > \frac{t}{n}}
	\end{equation*}
	\\
	Then, replacing $t$ with $\varepsilon n$, we get %Custom
	\begin{equation}
		\Pr{\abs{\frac{1}{n}\sum_{i=1}^{n}(X_i) - p} > \varepsilon} \leq 2e^{-2n^3\varepsilon^2}
	\end{equation} % TODO check binomial c.b.
	
	\ex If you want to know how many Facebook users have more than 100 friends, you can't simply count, but you can obtain a good approximation sampling a reasonable number of profiles.
	
	It's important to observe that the size of the sample we need in order to obtain a good approximation depends on the error we tolerate, not on the total population.
	
	There are many possible formulations of the Chernoff bound, another one that can be useful is the following:
	\begin{equation}\label{chernoff2}
	\Pr{\abs{X-E[X]} \geq \varepsilon \cdot n} \leq 2e^{-\frac{\varepsilon^{2}}{3}n}
	\end{equation}
	
	\ex Let be $E[X]=\frac{50}{100}$ and $\varepsilon=\frac{1}{100}$, the probability that $X \geq \frac{51}{100}$ or $X \leq \frac{49}{100}$ is less then or equal to $2e^{-\frac{1}{30000}n}$.
