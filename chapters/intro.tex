% !TEX spellcheck = en-US

\chapter{Introductory pills}

\section{Objectives of the course}
    
    The main objective of the course are:
    \begin{enumerate}
        \item Find efficient solutions to analyze social networks;
        \item Define models that describe users' behavior;
        \item Find ways to infer missing data in a social network.
    \end{enumerate}

    Our models have to adapt well to real cases but also to be understandable.
    
    Our algorithms have to be efficient: they need few memory and execution time, and they can easily be executed in parallel on more computers, using frameworks such as Pregel or Map-Reduce.


\section{Basic ideas}

    Here we enumerate some preliminary notes and formulas that should be already known and will be useful during the course:
    
    \begin{itemize}
        \item \textit{Note}: ``pick" and ``sample" are used interchangeably;
        \item We call \emph{node} an entity on the web that can be a blog, a website or a social network profile
        \item A useful property of logarithms:
            \begin{equation}\label{eq:log-prop}
                e^{\ln x} = x
            \end{equation}
        \item A recurrent inequality:
            \begin{equation}\label{eq:e-x}
                1-x \leq e^{-x}
            \end{equation}
        \item A helpful limit:
            \begin{equation}\label{eq:limit-1/e}
                \lim_{x \to +\infty} \left(1 - \frac{1}{x}\right)^x = \frac{1}{e}
            \end{equation}
        \item Let $E_i$ be mutually independent events, then
            \begin{equation}\label{eq:prob}
                \Pr{\bigcap_{i=0}^n E_i} \leq \prod_{i=0}^{n}\Pr{E_i}
            \end{equation}
    \end{itemize}
    
    Then we have some useful definitions.
    
    \begin{defn}[Expected value]
        $E[X]$ is called the \textit{expected value} because it means the midpoint where all outcomes, compounded with their weight, contribute to the equilibrium.
        
        \begin{qst}
            Can we still talk about expected value when the outcomes are not numeric? What could be a good bijection between a generic sample space $\Omega$ and $\nonneg$?
        \end{qst}
    \end{defn}
    
    \begin{defn}[Sampling]
        It is a simple algorithmic approach used for abstracting and estimating the percentage of elements of a certain type in a large set.
        
        \ex If there is a huge number of black and white marbles in a (very big) bowl and you'd like to know the ratio of black marbles, you can't count all of them, so you extract a sample of marbles many times, until you have sufficient confidence in the obtained result.
    \end{defn}
    
    \begin{defn}[Bayes' Theorem]
        Bayes’ theorem (or law or rule) describes the probability of an event, based on prior knowledge of conditions that might be related to the event.
        \begin{equation}\label{eq-bayes}
            \Pr{(A \cap B)} = \Pr{(A)} \cdot \Pr{(B | A)}
        \end{equation}
    \end{defn}


\subsection{Pregel framework}

    Brief description:
    \begin{itemize}
        \item Each node can communicate only with its neighbors;
        \item The computation si asynchronous;
        \item Each node can use few local memory (since there are many nodes);
        \item The nodes of a network (graph) are distributed on many computers, so it's important to group them based on community to reduce the number of communications among computers, and to accurately choose a threshold between the memory to assign to each node and the number of nodes we want to put into each single machine.
    \end{itemize}
    
    \begin{ex}[Find connected components of a graph using Pregel]
        Each node contains a unique number at the beginning (an ID), then every one send its ID to its neighbors, and when a node receives a new number replace its ID with the received one, if it's geater than its own. \\
        When the values stop changing every connected component has the same value.
        The maximum number of iterations is given by the diameter of the graph.
    \end{ex}


\subsection{Union bound}

	\begin{defn}[Union Bound]
        Let $A$ and $B$ be two events, then
        \begin{equation}\label{eq:union-bound}
            \Pr{A \vee B} \leq \Pr{A} + \Pr{B}
        \end{equation}
    \end{defn}

    \begin{proof}
        The theorem directly follows from the fact that probabilities are non-negative and from the following property:
        \begin{equation}\label{eq:prob-or}
            \Pr{A \vee B} = \Pr{A} + \Pr{B} - \Pr{A \wedge B}
        \end{equation}
    \end{proof}
    
	\begin{defn}[Generalized Union Bound]
        Let $E_i$ be a countable set of events, then
        \begin{equation}\label{eq:union-bound-gen}
            \Pr{\bigcup_{i=0}^n E_i} \leq \sum_{i=0}^{n}(\Pr{E_i})
        \end{equation}
    \end{defn}
    
	If the events are mutually disjoint, then the equality is strict.

	
\subsection{Markov inequality}
	
	Let $X$ be a non-negative random variable (RV) in a sample space $\Omega$, and $a$ an outcome in $\Omega$, then:
	
	\begin{equation}\label{eq:markov}
	   \Pr{X \geq a} \leq \frac{\mathbb{E}[X]}{a}
	\end{equation}
	
	\begin{proof} \
        \begin{flalign*}
            \mathbb{E}[X] &= \int_{0}^{\infty}x \Pr{X=x}dx& \\
            &\geq \int_{a}^{\infty}x\Pr{X=x}dx& \\
            &\geq \int_{a}^{\infty}a\Pr{X=x}dx& \\
            &= a\Pr{X \geq a}
        \end{flalign*}
    \end{proof}
	
	Informally, the Markov inequality let us tell something about an extreme event in a probability distribution of which we know only the expectation: if $\mathbb{E}[X]$ is small, then X is unlikely to be very large.
    
    Another way of claiming this inequality is the following:
    \begin{equation}\label{eq:markov2}
        \Pr{X \geq a \cdot \mathbb{E}[X]} \leq \frac{1}{a}
    \end{equation}
	
	More about this \href{https://youtu.be/vjYanZ1nsZg}{here} or \href{https://ocw.mit.edu/resources/res-6-012-introduction-to-probability-spring-2018/part-i-the-fundamentals/}{here}.


\subsection{Moment generating functions}
	
	Moments, in a mathematical-analytical sense, are ``quantitative measures" that describe characteristic of a shape. E.g.: a generic function's first moment is its slope (first derivative).\\
	Given a random variable $X$, then its moment generating function is
	\begin{equation}
	M_X(t) = e^{tX}, t \in \mathbb{R}
	\end{equation}
	
	To be continued... % TODO moment generating function
	
	Reminder! Power series of the $e^x$ function: $\displaystyle e^x = \sum_{i=0}^{\infty}\frac{x^n}{n!}$
