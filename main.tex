% !TEX TS-program = xelatex
% !TEX spellcheck = en-US

\documentclass{report}

% mathbb command, complement symbol and others
\usepackage{amssymb}

% More blackboard symbols
\usepackage{bbm}

% binom macro
\usepackage{amsmath}

% Apply paragraph spacing
\setlength{\parskip}{1em}

% Put all formulae in display style for fractions, binomial coefficients, etc.
\everymath{\displaystyle}

% Theorem symbols
\usepackage{amsthm}

%Some quick defs - sets
\newcommand{\nonneg}{\mathbb{N}}
\newcommand{\integer}{\mathbb{Z}}
\newcommand{\real}{\mathbb{R}}
\newcommand{\singleton}{\mathbbm{1}}
\newcommand{\binary}{\mathbbm{2}}
%Some quick defs - prob distributions
\newcommand{\probdist}{\mathcal{R}and}
\newcommand{\unifdist}{\mathcal{U}nif}
\newcommand{\binomdist}{\mathcal{B}inom}
\newcommand{\berdist}{\mathcal{B}er}
%Some quick defs - abbreviations
\newcommand{\iid}{\,\textsc{IID}}
\newcommand{\uar}{\,\textsc{UAR}}
%Some quick defs - similarities
\newcommand{\jaccsim}{\mathcal{J}acc}
\newcommand{\hammsim}{\mathcal{H}amm}
%Some quick defs - other
\newcommand{\permut}{\mathcal{P}erm}
\newcommand{\indep}{\perp\!\!\!\perp}

% Some redundant names
\newcommand{\coin}{\berdist}
\newcommand{\order}{\permut}

\title{Uniroma1(2018-19) - Social and Behavioural Networks}
\author{Andrea Proietto}
\date{October 2018}

%Questions:
% - Revise Chernoff-Hoeffding, can't get a definitive formula

\begin{document}

\maketitle

%\chapter{Preamble}
%Questi appunti utilizzano una notazione personale che sto portando avanti da anni, con l'obiettivo di essere il meno ambiguo possibile. Un caso comune è il seguente: per definire una funzione, utilizzo questa sintassi:

%\begin{equation}
%f \in A \to B : a \mapsto f(a)
%\end{equation}

%dove $f$ è la funzione, $A \to B$ è il \textbf{tipo} di funzione, specificato dal dominio $A$ e dal codominio $B$, ed è trattato come insieme di tutte le funzioni da $A$ a $B$, giustificando l'uso dell'operatore di appartenenza. La parte che segue i due punti definisce la funzione attraverso la notazione di mappatura.
%Ad esempio, l'espressione seguente:

%\begin{equation}
%d \in \mathbb{N} \to \mathbb{N} : n \mapsto 2\cdot n
%\end{equation}

%si leggerebbe: "$d$ è una funzione che va dai naturail ai naturali tale che ad n associa 2n"





\chapter{Introductive pills}
	
	%Note: "pick" and "sample" are used interchangeably
	
	%About E(X): It is called the "expected value" because it means the midpoint where all outcomes, compounded with their weight, contribute to the equilibrium
	
	%\textit{Question: Can we still talk about expected value when the outcomes are not numeric? What could be a good bijection between a generic Omega and N?}
	
	%Union bound: $\displaystyle \Pr(\bigcup_{i=0}^n E_i) \leq \sum_{i=0}^{n}(\Pr(E_i))$ ; if the events are mutually disjoint, then the equality is strict
	
	\
	
\subsection{Markov inequality}
	
	Let X be a nonnegative RV in $\Omega$, and an outcome $a \in \Omega$, then:
	
	\begin{equation}
	\Pr(X \geq a) \leq \frac{E(X)}{a}
	\end{equation}
	
	Proof: $ E(X) = \int_{0}^{\infty}x\Pr(X=x)dx \geq \int_{a}^{\infty}x\Pr(X = x)dx \geq \int_{a}^{\infty}a\Pr(X = x)dx = a\Pr(X \geq a)$
	
\subsection{Moment generating functions}
	
	A moment in a mathematical-analytical sense, are "quantitative measures" that describe characteristic of a shape. E.g.: a generic function's first moment is its slope (first derivative), 
	Given a a random variable X, then its moment generating function is the function $M_X(t) = e^{tX}, t \in \mathbb{R}$
	
	To be continued...
	
	Reminder! Power series of the $e^x$ function: $\displaystyle e^x = \sum_{i=0}^{\infty}\frac{x^n}{n!}$
	
	
\section{Chernoff bound}
	
	Let $X_1, \dots, X_n \in \coin(p) \iid$, or in other words, n independent coin throws of the same coin; then define X as $\sum_{i=1}^{n}(X_i)$. X is essentially a binomial RV: $X \in \binomdist(\binary^n, p)$, thus $E(X)=p\cdot n$
	
	So, given an error limit $\varepsilon$, we define the Chernoff bound as:
	
	%True definition
	$\displaystyle \Pr(X-E(X) \geq t)\leq e^{-2nt^2}$
	
	Which translates, in the binomial case to:
	
	$\displaystyle \Pr(|\sum_{i=1}^{n}(X_i) - p \cdot n| > t) = \Pr(|\frac{1}{n}\sum_{i=1}^{n}(X_i) - p| > \frac{t}{n})$

	%Custom
	Then, by defining $t := \varepsilon n$
	
	$\Pr(|\frac{1}{n}\sum_{i=1}^{n}(X_i) - p| > \varepsilon) \leq 2e^{-2n^3\varepsilon^2}$
	
\chapter{Locality Sensitive Hashing}
	
	The goal of hashing techniques is to reduce a big "object" to a small "signature" or "fingerprint". In general, what happens in locality sensitive hashing (or LSH) is to have some notion of similarity, and then define a "scheme" which computes it. The process of creating a scheme usually involves some sort of preprocessing step, and a function family which, by choosing one or another function according a probability distribution, statistically classifies the objects in the same way as the similarity function does.
	The bottom line of LSH schemes is: similar objects hash to similar values.
	
	Here are some common similarities:
	\begin{itemize}
	
\item \textbf{Jaccard similarity}: Given two sets of objects A and B, their Jaccard similarity is defined as follows:
	
	\begin{equation}
	\jaccsim(A, B) = \frac{|A\cap B|}{|A\cup B|}
	\end{equation}
	
\item \textbf{Hamming similarity}: Given two sets of objects A and B taken from a universal set U, their Hamming similarity is defined as follows:
	
	\begin{equation}
	\hammsim(A, B) = \frac{|A\cap B| + |\complement_U(A\cup B)|}{|U|}
	\end{equation}
	
	\end{itemize}
	
\section{A case study: Web-page indexing}
	
	A search engine crawls periodically the Internet and stores valuable information in its own index for search optimization purposes.
	
	An observation to make is the following: some kinds of documents, that are very similar to each other, are stored sparsely through the net; to save storage space, only one of a kind of document's info is stored in the index, whereas all others are linked to the first one, because of their similarity.
	
	To find a useful hashing scheme, A. Broder came up with an idea. First off, let us fix some definitions:
	\begin{itemize}
	\item $U$: the set of all words, i.e. the English vocabulary
	\item $U*$: the set of all strings composed of english words
	\end{itemize}
	
	The starting point is to treat web pages as strings:
	
	$T_1:$ \textit{``The black board is black''}
	
	$T_2:$ \textit{``The white board is white''}
	
	Then, let $distinct(T)$ returns the set of distinct words appearing in a string (Ref. Bag of Words model)
	
	So for example, by using the Jaccard similarity:
    
    \begin{equation}
    \jaccsim(distinct(T_1), distinct(T_2))= \frac{3}{5}
    \end{equation}
	
	Over a half: they look close. If we used the Hamming distance instead, we would (almost always) get a number very close to 1, because we're using a minuscule part of the universe set (in our case, the English dictionary), thus (almost) all words are absent from the sets.
	
	Now, our objective is to construct a scheme over web-pages that implement the Jaccard similarity. Our pre-processing step: Choose a permutation (or total ordering) $p \in \order(A \cup B) \uar$. To construct said order is a simple task:
	
	Algorithm (using u):
	\begin{verbatim}
	pi: empty sequence
	while u is not empty:
		pick a word w from u UAR and remove it from u
		append w to pi
	end
	return pi
	\end{verbatim}
	
	Proof of uniform choice: $\Pr(X = \pi) = \frac{1}{|u|}\cdot\frac{1}{|u|-1}\cdot\dots\cdot 1 = \frac{1}{|u|!} \Rightarrow X \in \unifdist(\permut(A \cup B))$
	
	From $p$, we define the hashing function as:
	
	\begin{equation}
	h_p \in \mathcal{P}(U) \to U : h_p(A) = min_p(A)
	\end{equation}
	
	In other words, we take the ``minimum'' in A according to the ordering specified by $p$. A simple but useful observation would be:
	
	\begin{equation}
	\forall A \subseteq U \Rightarrow h_p(A) \in A
	\end{equation}
	
	E.g.: $p = (black, the, is, white, board) \wedge A = \{black, board\} \Rightarrow h_p(A) = black$
	
	Thus, we can say that $A$ is similar to $B$ iff $h_p(A)=h_p(B)$. Recall that $A$ and $B$ are fixed, $p$ is the focus of this definition. What can be said about $\Pr(h_p(A)=h_p(B))$ ? Looking at a corresponding Venn diagram:
	
	\begin{itemize}
	\item $A \cap B = \emptyset \Rightarrow \Pr(h_p(A)=h_p(B)) = 0$; they have no words in common, so their hashes must be different, independently of the chosen order;
	\item $A = B \Rightarrow \Pr(h_p(A)=h_p(B)) = 1$; this time, all words are in common, so their hashes must coincide, again, independently of the chosen order
	\item Otherwise, since $p$ is chosen $\uar$, the probability that the hashes are equal has the same meaning of the probability of finding the lowest element of A and B in the intersection with respect of the union (and not in $U$ as a whole, as our previous observation suggests) which is the Jaccard similarity of A and B by its very definition: $\Pr(h_p(A)=h_p(B)) = \jaccsim(A, B)$
	\end{itemize}
	
	Possible question about third point: Why not respect to the universal set? because A and B will have hashes which, as we observed earlier, do not live outside the union: the union between A and B is our true set of outcomes when hashing either A or B.
	
	Now, if $h_p$ is evaluated only once over a given permutation, only a binary response can be obtained. In order to obtain the probability value without resorting to compute unions and intersections, we can repeat evaluation over different permutations; this can be regulated by the Chernoff-Hoeffding bound:
	
	Let $A, B \subseteq U$, and $X_{1 \dots n} \in \coin(p) \iid$, each defined over a distinct element of $\Pi \subseteq \mathcal{P}erm(U)$ such that $X_i \mapsto 1 \Leftrightarrow A \sim_{\pi_i} B, 0\ otherwise$, then:
	
	%TODO
	\begin{equation}
	\displaystyle \Pr\left(\left|avg_{i=1}^{n}(X_i) - \frac{\sum E(X)}{n}\right|<\varepsilon\right) = 
	\end{equation}
	
	%HAMMING!!
	
\section{LSH formalization}

    First let us focus around the hash function as an object: its true purpose in a scheme is to classify objects based on how much they ``look like'', whatever this means in the chosen similarity's terms. Therefore, in our theoretical analysis, the codomain of a hash function is not that important; what is important is how the function partitions its own domain, $U$. In a sense, we're interested only in the partitions of $U$ themselves, not in the functions that generate them.
    
	Why have we dealt with functions back then? Moving from a purely mathematical perspective to a more computational one, what is usually done for measuring similarities is sampling some object's characteristics, and observe how ``distant'', or else ``similar'' they are. This is done by means of some program; and programs are (oh so) easily associated with functions. The computational approach gives a more intuitive vision of the problem we're confronting ourselves with.
    
    Still, what could happen, is to have a couple of functions that map values into wildly different codomains, but partition $U$ in exactly the same way! And in our journey, we're just interested in classifying objects; so these kind of ``duplicate'' functions are, well, useless (unless we delve in complexity studies, but that's out of the scope).
    
	So, let us reform the foundations by taking as our core object a universe partition, instead of a universe-domained hash function. First, though, we need to formalize what a similarity is, and to get to a good definition, we have to carefully select them from their space $U^2 \to [0, 1]$. It should be noted that the codomain might very well be $\real$ itself, but to get some bearings we'll treat an image of 1 as a complete equivalence between two objects, and 0 for complete difference, with the interval expressing the degree of similarity.
	
	
	
	Let $U$ be a set, and $S \in U^2 \to [0, 1]$ a symmetric function; then $S$ is called a similarity over $U$.
	% Still incomplete, need the trineq on 1-S
	
	Tidbit: Let $f \in A^n \to B$, then f is \textbf{symmetric} iff \textit{argument order does not change the image} % COMMUTATIVITY? Kinda...
	
	% A scheme-similarity defines a partition over U
	A LSH scheme over U is a probability distribution over U's partitions
	
	%%%%%%%%%%%%%%%%%%%%
	
	Intuitive: a hash function's purpose is to map arguments that are very similar to the same value
	
	Hash function: $h \in U \to (*)$ such that the domain's representation is 'sensibly smaller' than U (h is NOT injective by this intuition - not true, injective functions are used as hash functions in pathological cases...)
	
	Insight: A hash function seems to complicate definitions more than a simple equivalence relation, but it models programs/algorithms more effectively, which is our focus here
	
	%Given a similarity $\phi$, a LSH scheme is a family of hash functions $H$, coupled with a prob. distribution $D$ over $H$ "such that, chosen a function $h$ from the family $H$ according to the prob.dist., satisfies the property $\Pr(h(a)=h(b)) = \phi(a,b) \forall a,b in U$"
	
	%Rewritten
	%Let $S \in U^2 \to [0, 1]$ be a similarity, and H be a RV over a family of hash functions over U, then H is a LSH scheme iff $\Pr(H(a)=H(b)) = S(a,b) \forall a,b in U$
	
	Brainlamp: I can extend the domain of H to the whole hashfunction class by setting the outsiders' probability to 0...
	%%%%%%%%%%%%%%%%%%%%%%%%%%%%%%%%%%%%
	
	
	Note: some partitions will never be a result of a hash function % Hah, not so sure...
	
	Other note: some form of transitivity must hold. (So yeah, we're dealing with an equivalence relation in the guise of a function with arbitrary codomain)
	
	\
	
	INSIGHT: Preprocess \& hash function (aka a scheme) determine the similarity function (most people attempt to do the reverse)
	
	\
	
	MAJOR INSIGHT: In the previous webpage example, we're not dealing with a single hashing function, but with a family of functions each built with its own word permutation: the scheme distributes over the permutations of the union!
	
	\
	
	Wrapup: A LSH scheme for a similarity S is a prob dist over U's partitions such that $\forall A, B \in  U \Rightarrow \Pr_\pi(A\sim_\pi B) = S(A, B) = \Pr_h(h(A)=h(B))$
	
	Challenge: Can we find a LSH scheme for an arbitrary S function? NO
	
	E.g. $U = \{a, b, c\}$
	$S \in U^2 \rightarrow [0, 1] : S(a, b) \mapsto 1, S(b, c) \mapsto 1, S(a, c) \mapsto 0$ % We're violating transitivity
	
	Translating into probabilities and using equality's transitivity, we obtain: $\Pr(h(a), h(c))=1$, which contradicts the third mapping.
	
	 
	
	
	\chapter{Next}
	
	More distances:
	
	Dice similarity
	\begin{equation}
	\displaystyle D(A, B) = \frac{|A\cap B|}{|A\cap B| + \frac{1}{2}|A\vartriangle B|}
	\end{equation}
	
	Anderberg similarity
	\begin{equation}
	\displaystyle D(A, B) = \frac{|A\cap B|}{|A\cap B| + 2|A\vartriangle B|}
	\end{equation}
	
	Generalizing Jaccard, Dice and Anderberg: 
	\begin{equation}
	\displaystyle S_\gamma(A, B) = \frac{|A\cap B|}{|A\cap B| + \gamma|A\vartriangle B|}
	\end{equation}
	
	\
	
	Important lemma (Charikar): if a similarity S admits a (LSH) scheme, then $1-(S)$ must satisfy the triangular inequality (trineq)
	
	Proof: %TODO SEE NOTES
	
	Afterthought: Similarities are actually defined in most cases as the inverses of measures, which in turn give (oh so surprisingly!) a notion of distance
	\
	
	By Charikar's lemma, we can prove that Dice's similarity cannot admit a LSH scheme
	
	Proof by counterexample: Assume $A=\{1\}, B=\{2\}, C=\{1, 2\}$, then use the trineq over the distances; %\textreferencemark
	
	Parameterizing this counterexample with $S_\gamma$, we obtain a bounds for gamma: $1 \leq \frac{2\gamma}{1 + \gamma} \Rightarrow \gamma \geq 1$
	
	\
	
	\subsection{Probability generating functions}
	
	Intuition: A probability generating function (PGF) is a \textbf{power series representation} of a given probability distribution
	
	Definition: Given a (discrete) RV X, its PGF is the function:
	
	$\displaystyle \mathcal{G}en_X(\alpha)= \sum_{x=0}^{|\Omega|}\Pr(X=x)\alpha^x$ (note that all outcomes appear by their probability)
	
	How to get back to pmf: $\displaystyle \Pr(X=x) = \frac{\mathcal{D}^x(\mathcal{G}en_X(0))}{x!}$
	
	\
	
	Theorem: If a similarity S admits a LSH and a given function f is a PGF, then f(S) admits a LSH
	
	Afterthought: Are we "applying" a probability distribution over a similarity (which in turn, since it is lshable, means it has a probdist over a subset of hashfunctions)?
	
	Proof: see below
	
	Consequence: Applying the theorem to the Jaccard similarity:
	
	%Our PGF is $\displaystyle f_\gamma(x) = \frac{x}{x+\gamma(1-x)} = $ %WRONG???
	
	Start with $\displaystyle f_\gamma(\alpha) = \sum_{x=1}^{\infty}\frac{(1-\frac{1}{\gamma})^x}{\gamma -1}\alpha^x$
	
	We have to demonstrate that the coefficients represent a probability distribution: $\displaystyle \sum_{i=1}^{\infty}\frac{(1-\frac{1}{\gamma})^i}{\gamma -1}=1 \Rightarrow \sum_{i=1}^{\infty}(1-\frac{1}{\gamma})^i=\gamma -1$
	
	Now: $\displaystyle \sum_{i=1}^{\infty}(1 - \frac{1}{\gamma})^i = (1 - \frac{1}{\gamma})\sum_{i=0}^{\infty}(1 - \frac{1}{\gamma})^i = (1 - \frac{1}{\gamma}) \frac{1}{1 - (1 - \frac{1}{\gamma})} = \frac{\gamma -1 }{\gamma}\frac{1}{1/\gamma} = \gamma -1$
	
	
	Apply PGF to Jaccard: $f_\gamma (J(A, B)) = $ ...algebretta insiemistica... $= \frac{\cap}{\cap + \gamma\vartriangle} = S_\gamma(A, B)$, which in turn is LSH-able
	
	
	\
	
	\section{181008}
	
	
	%%%%%%%%%%%%%%%%%%%%%%%%%%%%%%%%%%%%%%%%181008
	
	%% So it actually matches the previous wikipedian def
	Recap: Given a universe U, a function $S \in U^2 \to [0, 1]$ is said to be a LSH-able similarity iff exists prob distr over (a family/subset of) the hash functions in U, such that: 
	\begin{equation}
	\forall X, Y \in U\ \ \Pr_h(h(X)=h(Y)) = S(X, Y)
	\end{equation}
	
	
	Reprise: If a similarity S is LSHable and f is a PGF, then f(S) is LSHable
	
	Equivalent statement:
	\begin{equation}
	f(S) := T \in U^2 \to [0, 1] : \forall A, B \in U\ \ T(A, B) = f(S(A, B))
	\end{equation}
	
	%Afterthought: could we just demonstrate that probability distributions are composable? lshability could be just a carried-over property...
	
	
	%%%%%%%%%%%%%%%%%%%%%%%%%%%%%
	
	\
	
	Lemma (1): The similarity $O \in U^2 \to [0, 1] : (A, B) \mapsto 1$ admits a LSH
	Proof: Give prob. 1 to a constant function (duh!): $h \in U \to \mathbbm{1} A \mapsto 0 \forall A in U$
	
	Purpose: This will be the base case for theorem proof...
	
	%%%%%%%%
	
	\
	
	Lemma(2): If S and T similarities over U have a scheme, then $S \cdot T : (S \cdot T)(A, B) = S(A, B)\cdot T(A, B)$ has a scheme
	
	(I.E:). LSHability is preserved upon composition/multiplication
	
	Proof by construction (Algorithm): 
	
	sample hash functions for $S \cdot T$ as follows
	
	first, sample $h_S$ for S;
	then sample $h_T$ independently for T
	return the function $h : A \mapsto (h_S(A), h_T(A))$
	
	$\Pr_h(h(A)=h(B)) = \Pr_{h_S}(h_S(A)=h_S(B)) \cdot \Pr_{h_T}(h_T(A)=h_T(B)) = S(A, B) \cdot T(A, B) (\forall \{A, B\} \in \mathcal{P}_2(U))$by independency
	
	%%%%%%%%%%%%%%%%%%%%%%%%%%%%%%%%%%%%%
	
	Lemma(3): if S is LSHable then $\forall i \in \mathbb{N}$ $S^i$ is lshable
	
	proof by induction:
	
	base (Lemma1): i=0 and $S^0$=O OK
	
	ind: Use lemma 2 on $S^i$ and $S$ to obtain $S^{i+1}$; S has a scheme by def, $S^i$ has a scheme by induction hypothesis
	
	%%%%%%%%%%%%%%%%%%%%%%%%%%%%%%%%%%%%%%%%%%%%%%%%%%%%%%%
	
	lemma(4): if $p_0, ..., p_i, ... : \sum_{i=0}^{\infty}p_i=1 , and p_i\geq 0 \forall i$, and $S_0, ..., S_i, ...$ are lshable similarities, then $\sum_{i=0}^{\infty}p_iS_i$ is lshable
	
	scheme: first pick(sample, they are synonyms) i* at random from $\mathbb{N}$ with probability $p_0, ..., p_i, ...$
	then, sample h from the hash functions of $S_{i*}$
	
	$\Pr(h(A)=h(B))=\sum_{i=0}^{\infty}(p_i S_i(A, B))$
	
	$\Pr(h(A)=h(B))=\sum_{i=0}^{\infty}(\Pr(i=i*)\Pr_h(h(A)=h(B) | i=i*))$, $\Pr(i=i*) = p_i$, $hahb|i=i* \to S_i(A, B)$
	
	%%%%%%%%%%%%%%%%%%%%%%%%%%%%%%%%%%%%%%%%%%%%%%%%%%%%
	
	Final proof: sum of $p_iS^i$ has a scheme
	
	L3: $S^i$ has a scheme
	
	L4: the sum is lshable, proven
	
	%%%%%%%%%%%%%%%%%%%%%%%
	
	Example of a pgf: $sum(2^-1 times x^i)$
	
	
	----------------------------------------------------------------??sorensen dice??cosine similarity?? inner product??johnson-lindenstrauss??
	
	a sketch is n instance of a pgf which can be used to implement other similarities NONONONONOO
	
	
	%%%%%%%%%%%J
	
	PGF is an approach for making schemes for similarities from other schemes
	
	\section{181010}
	
	Let f a PGF, $\alpha \in [0, 1]$, ...?
	
	%&
	$\alpha f = \alpha f(S)\ |\ (1 - \alpha)T$
	%&
	
	$T \in U^2 \to [0, 1] : \forall {t, t'} \in \mathcal{P}_2(U) T(t, t') = 0$
	
	for the 1 case we wanted a nbanal partition, now with 0 we want a punctual partition, so we need a hash function that assigns a distinct value to each argument. (You can actually use the identity)
	
	Not a good scheme, because we're not shrinking data
	
	GOTO %&
	
	\subsubsection{Approximation examples}
	
	Consider $S_\gamma$: $\gamma \geq 1 \Leftrightarrow S_\gamma$ is LSH-able
	
	Focus on $\gamma < 1$
	
	Definition: "Distortion of a similarity"
	
	Let S be a similarity, then its distortion is "the minimum*(meaning inferior extremum) delta geq 1 : exists LSHABLE S' (forall {A, B} in P2U, $1/\delta \cdot S(A, B) \leq S'(A, B) \leq S(A, B)$)"
	
	delta tends to 1 -> S is lshable
	
	using jaccard to approximate sgamma we wold obtain 1/gamma
	
	delta is the approximation factor
	
	%%%%%%%%%%%%%%%%%%
	
	Centerlemma: Let S be a LSHable similarity $ : \exists \chi \subseteq U : \forall \{x, x'\} \in \mathcal{P}_2(\chi) S(x, x')=0$, then
	
	$\forall y \in U avg_{x \in \chi}(S(X, Y) \leq) \frac{1}{|\chi|}$; (it trivially follows that $\exists x* \in \chi : S(x*, Y \leq \frac{1}{|\chi|})$)
	
	Proof: (Fix $y \in U$) If the hash function h has positive probability in the (chosen) lsh for S, then $\forall {x, x'} \in \mathcal{P}_2(\chi) (h(x)\neq h(x'))$
	
	[$\chi$ is actually a (possibly incomplete) section of the partition of U induced by h]
	
	thus, forall hash functions with positive probability, there can exist at most one x in calx st hx=hy (transitivity of equality)
	
	$\sum_{x \in \chi}S(x, y) = \sum_{x \in \chi}\Pr_h(h(x)=h(y)) = \sum_{x \in \chi}\sum_{h}\Pr(h\ is\ chosen)[h(x)=h(y)]$
	
	IMPORTANT: The brackets here are a boolean evaluation operator
	
	$= \sum pr h is chosen \sum eval hx = hy \leq \sum_h \Pr(h\ is\ chosen) \leq 1$
	
	Thus, $\sum_{x \in \chi}S(x, y) \leq 1 \Rightarrow avg(S(X, Y)) \leq \frac{1}{|\chi|}$, proven
	
	
	%%%%%%%%%%%%%%%%%%%%%%%%%%%%%%%%%%%%%%%%%%%%%%%%%%%%%%
	
	
	$S:= S_\gamma , 0 < \gamma < 1, U=2^{[n]}={S|S\subseteq [n]}$ %insieme delle parti?
	
	$\chi := \mathcal{P}_1([n])$
	$y = [n]$
	
	 - let us assume that T finitely distorts sgamma, and T is lshable
	then $T({},{}) = 0 forall {i, j} in P_2(mathbbm(n))$
	
	 - $\exists \{i\} in \chi : T(\{i\}, [n]) \leq \frac{1}{|\chi|} = \frac{1}{n}$
	 
	 $S_\gamma(\{i\}, [n]) = \frac{1}{1 + \gamma(n-1)} = \frac{1}{\gamma n + (1-\gamma)}$
	 
	 ...
	 zenterlemma application
	 ...
	
	
	%%%%%%%%%%%%%%%%%%%%%%%%%%%%%%%%%%%%%%%%%%%%%%%%%%%%%%%%%%%%
	
	$ U = {}$
	
	
	\chapter{181015}
	
	First lesson reprise - Flow of memes
	
	spread of memes/viruses from a website to another
	
	so, we get traces
	
	goal is to reconstruct graph from traces
	
	Trace spreading model (from virology)
	
	source chosen UAR and edge traversal time chosen by (usually) Exp(lambda)
	trace starts from source
	
	obs. the first two nodes of a trace are connected
	
	pr a source is chosen: 1/n
	pr the edge's other endpoint: 1/deg(source) because of how traces work
	
	n numero di tracce: $pr = (1-1/deltan)^(3deltanlogn) = e^(-3lnn) simeq 1/n^3$
	
	
	\subsection{another information spreading process}
	
	NPR chain letter
	
	important observations: first name is fake, the others are plausibly real
	
	asks to add own's name, and then forward to all friends
	
	%%%%%%%%%%%%%%%%%
	
	someone breaks the rules, and decides to publish their copy of the mail
	
	those published mails reveal a subtree of the whole chainletter tree
	
	revealed tree MUCH smaller, and extended in-depth
	
	%%%%%%%%%%%%%%%%%
	
	Let T be a tree, each node has a chance of being "exposed" by a probability p, and if so, reveals all ancestors
	
	estimate the size of the tree
	
	ALGO
	throw a coin on all the nodes of the chain tree, don't throw if node is parent of a revealed node (no need) := special nodes
	
	special nodes are all iid from exposed ones; delta = n of special nodes/special nodes; revealed nodes = delta dot n (it's the expected value!!) -> from there estimate n
	
	delta is the actual exposure probability
	
	the two estimates can go wrong...
	
	examples in nature suggest some bounds
	
	%%%%%%%%%%%%%%%%%%%%%%%
	
	Theorem: the algorithm correctly estimates n if $n > omega(max(1/delta^2, k/delta))$
	
		
		IMPORTANT: Estimates can be redone, but reconstructing another revealed tree is impractical, so we're estimating a characteristic of  a tree by observing a sample revealed one and making estimates
	
	%%%%%%%%%%%%%%%%%%%%%%%
	Single-child fraction
	
	Fix a bound: a node's maximum degree must be lesser than k
	
	partition unknown tree into subforests, each subforest has 1/delta nodes and median height $omega(log_(k-1)^delta^-1)$
	
	by exposing a node in a subforest's lower half, we're exposing a number of nodes equal to the sf's median height
	
	
	
	%%181017%%%%%%%%%%%%%%%%%%
	
	Insight: $\delta$ is unknown too
	
	Recap of revealed subtrees
	
	Exposure prob: $\delta > 0$
	
	$E(\# exposed nodes) = \delta n$
	
	\
	
	Let $x \in RV$ be the set of nodes of the revealed tree. Let $Y \in Coin(p)$, where $Y_v = 1$ if v is exposed, 0 otherwise.
	
	$E(Y_v) = \delta$
	
	Let $ Y = \sum_{v \in X} Y_v \Rightarrow E(Y)= \delta |X|$
	
	OUTPUT: $\hat\delta = \frac{Y}{|X|}$
	
	Chernoff bound is applicable, the Ys are IID
	
	Note: we can do the first substitution because of linearity (?)
	
	"Multiplicative approximation": $\displaystyle \Pr(|\hat\delta - \delta| \geq \varepsilon \delta) = \Pr(|\frac{Y}{|X|} - \delta| \geq \varepsilon \delta) = \Pr(|Y - \delta |X|| \geq \varepsilon \delta |X|) = \Pr(|Y - E(Y)| \geq \varepsilon E(Y)) \leq 2e^{-\frac{\varepsilon^2}{3}E(Y)}, \forall \varepsilon \in (0, 1)$
	
	This is the Multiplicative Chernoff Bound: Let y1 ... yn IID Coins, then (see above, last inequality)
	
	we need E(y) to be large in order to obtain a useful bound, for E close to 1, the bound itself goes over 1, becoming useless
	
	$\displaystyle 2e^{-\frac{\varepsilon^2}{3}E(Y)} = 2e^{-\frac{\varepsilon^2}{3}\delta|X|}$
	
	If $\displaystyle |X| \geq \frac{3}{\varepsilon^2\delta}\ln2/\eta$, then $\displaystyle 2e^{-\frac{\varepsilon^2}{3}\delta|X|} \leq 2e^{-\ln2/\eta} = 2\eta/2 = \eta$
	
	$Y' = \sum_{v \in T}Y_v \Rightarrow E(Y') = \delta n$
	
	$\Pr(Chernoff\ on\ Y') \leq 2e^{-\frac{\varepsilon^2}{3}\delta n}$
	
	Using the bound over $|X|$, then $\leq \eta$
	
	Final output: $\displaystyle \frac{Y'}{\mathaccent 1 \delta } (\leq  \frac{(1 + \varepsilon)\delta n}{(1 - \varepsilon)\delta} = (1+O(\varepsilon))n )$
	
	Lower bound inverts the sign of bigOepsilon
	
	Insight: $\displaystyle \frac{Y'}{\mathaccent 1 \delta } = \frac{Y'}{Y}|X|$
	
	
	
	"An additive approximation is useless"
	
	
	goodnes sof alogrithm:
	
	 - nodes of unknown tree less than delta: revealed tree =0 almost surely
	 - tree is a star -> almost surely get a bad approximation
	 
	what characteristics should the unknown tree have?
	
	Claim: If the number of internal nodes of the revealed tree is at least 3/(epsilonSQRdelta)ln(2/eta) then our guess \^n is going to satisfy (1-oe)n <= \^n <= (1+oe)n with probability at least 1-2eta
	
	the three greek letters can change the goodness of our result
	
	Next goal: find the probability that a node of the unknown tree will be an internal node of the revealed tree
	
	Assumption: The unknown tree is such that none of its nodes has more than K children.
	%That actually has lots of applications! makes much sense
	
	Let $v$ be a node of the unknown tree with K children, then $\Pr($at least 1 children in K is exposed$) \geq 1-\varepsilon^{-1}min(1, \delta K_v)$
	
	$K_v$: expected number of children revealed
	
	Let $Z_v \in Coin(p)$ where it is 1 if at least one child of v is exposed
	
	$\displaystyle E(Z_v) = 1 - (1 - \delta)^{K_v} = 1 - ((1 - \delta)^{1/\delta})^\delta{K_v}$
	
	%using e's limit definition
	$\lim\limits_{\varepsilon \to 0^+}(1-\varepsilon)^{1/\varepsilon} = 1/e$
	
	$1 - ((1 - \delta)^{1/\delta})^\delta{K_v} \geq 1- e^{-\delta K_v}$
	
	
	\begin{itemize}
		\item if deltakv geq 1 then E(Zv) geq 1-1/e
		\item else , geometric intuition... ?????????? ... E(Zv) geq 1-ePOW(-deltakv) geq deltakv(1 - 1/e)
	\end{itemize}
	Lemma is proven
	
	
	\
	
	Former insight: for revelation we care about children, not all of the descendants
	
	By summing all children of all nodes, we get all the edges -> n-1
	% THis is used elsewhere...
	
	\
	
	Lemma: Let Z be the set of nodes of which at least 1 child is exposed; then $\Pr(|Z| \geq 1/2(1-1/e)\min(K^{-1}, \delta)(n-1))\geq 1-e^{-\Theta(n\min(1/K, \delta))}$
	
	Proof: Let I be the set of internal nodes of T; Let D subset I contain nodes with at least 1/delta children; $E(|Z|) = \sum_{v \in I}E(Z_v) \geq (1 - 1/e)\min(1, K_v\delta) = (...)(|D| + \delta\sum_{v \in I-D}K_v)$

	obs1: sumkvforvinI = n-1
	
	obs2: $|D| \geq \frac{\sum_{v \in D}K_v}{|K|}$
	
	then : $(1 - 1/e)(|D| + \delta\sum_{v \in I-D}K_v) \geq (1 - 1/e)(\sum K_v/K + \delta\sum_{v \in I-D}K_v) \geq (...)min(1/K, \delta)\sum = (...)min(...)(n-1)$
	
	Proven
	
	\
	
	see emergency notes!!
	
	\section{181022}
	
	-combinatoric optimization-
	
	graph independent set - impractical example
	
	An implicit opt problem.
	
	unknown - partly known function, exp in the codomain
	
	ex: describe SAT by means of truth tables - rows are exponential in variable number
	
	"Modular set functions"
	
	given $X=[n] f \in 2^X \to \mathbb{R}$
	
	$n=3, w_1 = 1, w_2 = 5, w_3 = 1, f(S)=\sum_{i \in S}$
	
	max  set: all domain
	
	max set w neg values: exclude neg values
	
	se ho un oracolo che mi calcola f(s), ed io non conosco f, posso interrogare l'oracolo sui singoletti, e poi applico le logiche precedenti
	
	(Matroid) Ex tutte parti con al più k elementi (a partire da U)
	
	"Ad placement"
	arriva un utente sul sito tipo Google, scegli quali ad mostrare in modo da guadagnare il più possibile
	
	modello Independent Cascade Model
	ogni ad rappresentata da una coppia $AD_i = (v_i, p_i)$ accordo tra pubblicità e Google, $v_i \in \$, p_i$ probobilità di click
	
	$Ad_{samsung} : (1\$, 0.1) \Rightarrow E[X_i] = 1 \cdot 0.1$
	
	Strategy: put highest expectation ads on top
	
	third value,: satisfaction factor (or the will to look other ads)
	
	how to optimize f in ordder to maximize it
	
	caso particolare "submodular set function" - analogo discreto delle funzioni convesse - can be misleading
	
	given X, f is submodular iff $\forall S, T \in X, f(S)+f(T) \geq f(S\cap T) + f(S\cup T)$
	
	modular is submodular: $\forall S, T \in X, f(S)+f(T) = f(S\cap T) + f(S\cup T)$

	quanti bit servono per rappresentare tale funzione?
	
	LAPSE
	
	COVERAGE (function, submodular): $X=\{S_1, S_2...S_m\}, S_i \subseteq [n]$
	
	$Y\subseteq X : c(Y) = |\bigcup_{S_i \in Y}S_i|$
	
	
	$s(Y) = \sum_{S_i \in Y} |S_i|$
	
	
	proof that f (or c?) is submodular
	base case n=1: couple S T can assume 4 combinations: $
	case s, t = \{1\} \Rightarrow union an intersection f(\cdot)= 1	
	1 in T, 1 not in S then 1 in union, 1 not in intersection
	dual is dual
	s, t = \emptyset
	$
	
	induction: split a coverage function into two coverage functions: $c(Y) = |\bigcup_{S_i \in Y}S_i| = |(\bigcup_{S_i \in Y}S_i)-\{n+1\}| + |(\bigcup_{S_i \in Y}S_i) \cap \{n+1\}|$
	
	Nemauser-W
	
	If f is submodular nonnegative and monotone, it can be approx to $max_\{|S|=K\}(f(S))$ to a factor of $1- \frac{1}{e}$
	
	represent modulars as points of convex multidimensional functions
	
	
	"maxcover" data una classe di sottoinsiemi di X,. S-1, si trovino k insiemiinX tail che , insieme, coprano il massinmo numero di elementi
	
	
	Before proving, ALGORITHM (greedy), parameterized by f, X and k: trovare k sottoinsiemi di X che massimizzano f
	
	$
	S_0 <- \emptyset
	for i=1 -> k
		let x_i = maximizer of f(\{x_i\} \cup S_{i-1})
		S_i <- \{x_i\} \cup S_{i-1}		
	return S_k
	$
	
	again before proving, observation: diminishing returns (formalization)
	
	$\delta(x|S) = f(\{x\}\cup S) - f(S)$ %always positif if f is monotone
	
	let f be submodular, $B \subseteq A, x \notin B \Rightarrow \Delta(x|A) \geq \Delta(x|B)$ % an iff can be demonstrated
	
	submodular functions exhibit diminishing returns property
	
	Now: $S=A\cup\{x\}, T=B \Rightarrow f(A\cup\{x\})+f(B)\geq f(A)+f(B\cup\{x\}) \Rightarrow f(A\cup\{x\})-f(A)\geq f(B\cup\{x\})-f(B) \Rightarrow \Delta(x|A) \geq \Delta(x|B)$
	
	proof of approximation:
	
	take an optimal solution $S*=\{x_1^*, ..., x_k^*\}$
	
	$\forall i, f(S^*) \leq f(S^* \cup S_i)$ %per monotonia di f
	
	$ = ... f(S_i) + \sum_{j=1}^{k}\Delta(x_j^*|S_i\cup \{x_1^*, ..., x_{j-1}^*\}) \leq (bydiminishingreturns) f(S_i) + \sum_{j=1}^{k}\Delta(x_j^*|S_i) -> \leq (bygreediness) f(S_i) + \sum_{j=1}^{k}\Delta(x_{i+1}|S_i) = f(S_i) + k(f(S_i \cup \{x_{i+1}\})-f(S_i)) = kf(S_{i+1}) - (k-1)f(S_i)$
	
	
	So: $f(S^*) - f(S_i) \leq k(f(S_{i+1} - f(S_i)))$
	
	Def: $\delta_i = f(S^*) - f(S_i)$
	
	Then: $\delta_i \leq k(\delta_i - \delta{i+1}) \Rightarrow k\delta_{i+1} \leq (k-1)\delta_i \Rightarrow \delta_{i+1} \leq (1-\frac{1}{k})\delta_i$
	
	$\delta_0 = f(S^*)-f(S_0) \leq f(S^*)$ %by nonnegativity
	
	
	.
	.
	.
	
	going to definition of e by its limit form $(1-1/k)^k$
	
	
	\section{181024}
	
	recap of yesterday:
	
	given a ground set, and $f \in 2^X \to \mathbb{}$, then f is modular iff $\forall S, T \subseteq X (f(S)+f(T) \geq f(s\cup t) + f(s \cap t))$
	
	E.g.: coverage functions, X is a set system $c(X) = \left| \bigcup_{S \in X} S \right|$

	"given a subset of sets, the coverage value would be the cardinality of the union"
	
	claim: coverage function is submodular
	proof: $c(S)+c(T) \geq c(s\cup t) + c(s \cap t)$
	
	proven case-by-case in the scope of where the elements are inside the subsets
	
	\
	
	if $f_1, ..., f_k \in 2^X \to \mathbb{R} are submodular, \alpha_1, ..., \alpha_k \geq 0 \Rightarrow \sum_{i=1}^{k}\alpha_i f_i is submodular$
	
	proof: the i-th case is easily proven, then sum up; nonnegativity is essential here
	
	\
	
	algorithm for optimization (?): $GREEDY_f(X, K)$:
	$
	S_0 <- \emptyset
	for i=1 -> k
		let x_i = maximizer of f(\{x_i\} \cup S_{i-1})
		S_i <- \{x_i\} \cup S_{i-1}		
	return S_k
	$
	
	if f is submodular, nonnegative and monotone, then $\displaystyle f(S_K)\geq(1-1/e)max_{S \in \binom{X}{K}}f(S)$
	
	\
	
	another algorithm: $GREEDY_{\varepsilon, f}(X, K)$:
	$
	S_0 <- \emptyset
	for i=1 -> k
		%let x_i be : f(S_{i-1} \cup \{x_i\}) \geq (1-\varepsilon)max_{x \in X} f(S_{i-1} \cup \{x\}) %finding the maximizer
		let x_i be : f(S_{i-1} \cup \{x_i\}) - f(S_{i-1}) \geq (1-\varepsilon)max_{x \in X} (f(S_{i-1} \cup \{x\}) - f(S_{i-1})) %finding the maximum marginal return
		S_i <- \{x_i\} \cup S_{i-1}		
	return S_k
	$
	
	the difference is: we don't need to find the exact maximizer, but something (1-epsilon) close to it; and it may be MUCH easier to find such something instead of the actual maximizer.
	
	\
	
	the core of the discussion is approximating a (usually impracticable to compute) function to a (polynomial) one
	
	Tidbit: proving correctness of this algorithm proves the previous one
	
	Trail: KKT model
	
	Hint(!): We don't need to find the maximum in this algorithm (otherwise, we are actually in the former case, why not use it directly?)
	
	\
	
	Theorem: ($\forall \varepsilon \in [0, 1)$), if f is submodular, nonnegative and monotone, then $f(S_k) \geq (1-\varepsilon)(1-1/e)max_{S \in \binom{X}{K}}f(S)$

	Proof: $S^* = argmax\ f(S), S \in \binom{X}{K}$
	
	$\forall i \in \mathbbm{k}$
	
	$f(S^*) \leq f(S_i) + \sum_{j=1}^{k}(f(S_i \cup\{x_1^*, ..., x_j^*\})-f(S_i \cup\{x_1^*, ..., x_{j-1}^*\}))$
	
	[defining diminishing returns (or marginal increase of {1} having {2})]
	
	$\to = f(S_i)+ \sum_{j=1}^{k}\Delta(x_j^* | S_i \cup \{x_1^*, ..., x_{j-1}^*\})$
	
	Because of submodularity, $\to \leq \sum_{j=1}^{k}\Delta(x_j^* | S_i)$
	
	-brainlapse-
	
	...at the end of the day $f(S^*) - f (S_i) \leq \frac{k}{1-\varepsilon}\Delta(x_{i+1}| S_i)$
	
	def $\delta_i = f(S^*) - f(S_i)$
	
	$\delta_{i+1} \leq \delta_i(1-\frac{1-\varepsilon}{k})$
	
	-SIGSEGV-
	
	see previous day to get all deltas from 0 to k
	
	then $f(S_k) \geq (1-(1-\frac{1-\varepsilon}{k})^k)f(S^*)=(1-((1-\frac{1-\varepsilon}{k})^{\frac{k}{1-\varepsilon}})^{1-\varepsilon})f(S^*) \geq (1-(1/e)^{1-\varepsilon})f(S^*)$
	
	more algebretta, use exponential convexity, get $f(S_k) \geq (1-e^{\varepsilon-1})f(S^*)$, over and out
	
	\
	
	Application: Kempe-Kleinberg-Tardos
	
	


\end{document}
